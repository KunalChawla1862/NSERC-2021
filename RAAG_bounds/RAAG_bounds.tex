\documentclass[12pt]{article}

\usepackage{amsmath, amssymb, amsthm, bbm, mathtools}
\newtheorem{defin}{Definition}
\newtheorem{thm}{Theorem}
\newtheorem{lem}{Lemma}
\newtheorem{example}{Example}
\newtheorem{prob}{Problem}
\newtheorem{sol}{Solution}

\usepackage{tcolorbox}
\usepackage{import}
\usepackage{pdfpages}
\usepackage{transparent}
\usepackage{xcolor}
\usepackage{lscape}
\usepackage{fullpage}
\usepackage{hyperref}

\newtcolorbox[auto counter]{problem}[2][]{ colback=red!5!white,colframe=black!75!white,fonttitle=\bfseries, title=Problem~\thetcbcounter: #2,#1}

\newcommand{\incfig}[2][1]{%
    \def\svgwidth{#1\columnwidth}
    \import{./figures/}{#2.pdf_tex}
}

\pdfsuppresswarningpagegroup=1

\begin{document}
\begin{defin}[Piling]
	Let $ \Gamma(G) $ be a RAAG with canonical generators\footnote{here $ v _{1}, ..., v _{n} $ are the vertices of the graph $ G $} $ v _{1}, ..., v _{n} $. Given a word $ w$ in the generators a piling $ \Pi (w) $ is a tuple of $ n $ strings in $ \{ 0,+,- \} ^{*} $ defined inductively as follows:
	\begin{itemize} 
		\item $ \Pi (e) $ is a tuple of empty strings.
		\item Given $ \Pi(w) $, $ \Pi (wv _{i}) $ is defined as follows: 

			\begin{enumerate} 
				\item If the $ i $th string is empty, ends in `0', or ends in `$+$', then $ \Pi(wv _{i}) $ is obtained from $ \Pi(w) $ by appending a `$ + $' to the $ i $th string, and appending a $ 0 $ to the $ j $th string for all $ j $ such that $ v _{j} $ and $ v _{i} $ don't commute
				\item If the $ i $th string ends in a `$ - $', then this character is removed, as well as the final `0' on each $ j $th string where $ v _{j} $ and $ v _{i} $ do not commute. \footnote{Since the $ i $th string ends in a ` $ - $', then the $ j $th string ends in a 0 if $ v _{j} $ and $ v _{i} $ don't commute.}
				  \end{enumerate} 
			  \item The piling $ \Pi(wv _{i} ^{-1})  $ is obtained likewise, by swapping the roles of `$ + $' and `$ - $'.
	 \end{itemize}
	  \end{defin}

\begin{defin} 
	Given an element $ g \in \Gamma(G) $, the terminal clique $ term(g) $ is the set of vertices $ v _{i} $ in the graph $ G $ such that the $ i $th string in the piling $ \Pi(g) $ ends in a `$ + $' or `$ - $'. The initial clique $ init(g) $ is the terminal clique of $ g ^{-1} $. 
	  \end{defin}

	  Given two pilings $ \Pi(g) $ and $ \Pi(h) $, the concatenation $ \Pi(g)\Pi(h) $ is defined componentwise by concatenating the $ i $th strings of each tuple for all $ 1 \leq i \leq n $.
\begin{lem} 
	Let $ g,h$ be words in the RAAG $ \Gamma(G) $ such that $ term(g) \cap init(h) = \varnothing $. Then $ \Pi(gh) = \Pi(g)\Pi(h) $. 
	  \end{lem}
	  \begin{proof}
		  We induct on the length of $ h $. First we observe that $ h $ can be represented by some reduced word in the generators, in the sense that it contains no subword of the form $ v _{i} x v _{i} ^{-1} $ where $ v _{i} $ commutes with all letters of $ x $. Indeed, if such a subword appears, we can apply our relations to replace it by $ x $. As each substitution strictly decreases the length of the word, we can eliminate all such occurences by finitely many moves.

		  Now write $ h $ in such a form and suppose that the word begins with $ v _{i} ^{k} $ or $ v _{i} ^{-k} $ where $ k $ is maximal (i.e. the word does not have $ v _{i} ^{k+1} $ or $ v _{i} ^{-(k+1)} $ as a prefix). Without loss of generality, suppose it starts with $ v _{i} ^{k} $ (the $ v _{i} ^{-k} $ case follows from the same argument).

		  Let us consider the piling of $ \Pi(gv _{i} ^{k}) $ -- we claim that $ \Pi(g v _{i} ^{k}) = \Pi(g)\Pi(v _{i} ^{k}) $. Since $ h $ can be represented by a reduced word beginning with $ v _{i} $, then the $ i $th string in the piling $ \Pi(h ^{-1}) $ ends in a `+' or `-', hence $ v _{i} $ is in the initial clique of $ h $. Therefore $ v _{i} $ is not in the terminal clique of $ g $. Then the $ i $th string in $ \Pi(g) $ is either empty or ends in a `$ 0 $'. Hence the piling $ \Pi(gv _{i} ^{k}) $ is obtained from $ \Pi(g) $ by appending $ k $ `$ + $'s and appending $ k $ `$ 0 $''s to all $ j $th strings where $ v _{j} $ does not commute with $ v _{i} $. This is precisely the concatenation $ \Pi(g)\Pi(v _{i} ^{k}) $.

		  Now let $ g' = gv _{i} ^{k} $ and let $ h' = v _{i} ^{-k} h $. As we chose $ k $ to be maximal, then $ v _{i} $ is not in the initial clique of $ h' $. Indeed, as the only other occurence of $ v _{i} ^{\pm} $ in the chosen word for $ h' $ occurs after some $ v _{j} ^{\pm}  $ not commuting with $ v _{i} $, then the corresponding word for $ (h') ^{-1} $ has no occurences of $ v _{i} ^{\mp} $ after $ v _{j} ^{\mp} $. Hence the $ i $th string in $ \Pi((h') ^{-1}) $ ends in a `0'.

		  Also observe that the terminal clique of $ g' $, which consists of all $ v _{i} $ such that $ \Pi(g') $ the $ i $th string ends in a `$ + $' or `$ - $', is contained in $ term(g) \cup \{v _{i}\} $. Hence $ term(g') \cap init(h') = \varnothing$. As $ |h'| $ is strictly less than $ |h| $, we know by our induction hypothesis that $ \Pi(g')\Pi(h') = \Pi(g'h') $.

		  Since $ |h'| < |h| $ and $ v _{i} $ is not in the initial clique of $ h' $, we can apply our induction hypothesis again to get $ \Pi(v _{i} ^{k})\Pi(h') = \Pi(v _{i} ^{k} h') = \Pi(h) $. Putting this together, we get \begin{align*} 
		  \Pi(gh) &= \Pi(g'h') \\
			     &= \Pi(g')\Pi(h') \\
			     &= \Pi(gv _{i} ^{k}) \Pi(h')\\
			     &= \Pi(g)\Pi(v _{i} ^{k})\Pi(h') \\
			     &= \Pi(g)\Pi(h).
	    \end{align*}
	  \end{proof}

\newpage

\section{Main argument}
Let $ d > 14 $ and consider a RAAG $ \Gamma(G) $ with connection graph a cycle with $ d $ vertices, labelled $ a _{1}, ..., a _{d} $. That is to say, $ \Gamma(G) $ is presented as \[ \langle a _{1}, ..., a _{d}| [a _{i}, a _{i+1}] = e\rangle  \] where the indices are taken modulo $ d $. Let $ \mu _{S} $ be the uniform measure on $ \{ a _{1} ^{\pm }, ..., a _{d} ^{\pm} \} $. Further suppose that $ \nu $ is a measure with $ \nu (e) = 0 $. Consider a random walk $ \mu _{S} * \nu $. In other words, our random walk is given by $ Z _{n} = s _{1} w _{1} ... s _{n}w _{n} $. 

We want to show that there exists some $ \kappa > 0 $ such that \[ \mathbb{P} (|Z _{n}| \leq \kappa n) \leq e ^{-\kappa n},\] where $ |Z _{n}| $ denotes the word length of $ Z _{n} $ with respect to the symmetric generating set given above.

To prove this, we condition on the $ w _{n} $'s and keep the randomness coming from the $ s _{n}$'s. We find a working $ \kappa $ that is independent of our conditioning, so this will prove the claim.


\begin{defin} 
	Say that a time $ k $ is pivotal with respect to $ n $ if $ |Z _{k}|, |Z _{k} s _{k}|, |Z _{k}s_k w _{k}| $ is strictly increasing, and if the piling for $ Z _{k} $ is a prefix of the piling for $ Z _{j} $ for all $ k \leq j \leq n $.
	  \end{defin}

	  Denote by $ P _{n} $ the set of pivotal times w.r.t $ n $ and let $ A _{n} = |P _{n}| $. We have $ |Z _{n}| \geq A _{n} $. We want to show that there exists some $ \kappa > 0 $ such that $ \mathbb{P} (A _{n} \leq \kappa n) \leq e ^{-\kappa n} $. To do this, we show that $ A _{n+1} $ stochastically dominates\footnote{in the sense that for any $ i $ we have $ \mathbb{P} (A _{n+1} \geq i) \geq \mathbb{P} (A _{n} + U \geq i)$} $ A _{n} + U _{n} $ where $ U _{n} $'s are i.i.d with positive expectation and exponential tail.

	  To discuss how the choices of $ s _{i} $ affect the number of pivotal times, we use the notion of a `piling' for an element of a RAAG (taken from ``\href{https://arxiv.org/pdf/0802.1771.pdf}{`The conjugacy problem in Right-Angled Artin groups and their subgroups}" by Crisp, Godelle, and West). We don't discuss them here, but we note that the authors prove that to each word in the RAAG there is a well-defined clique in the connection graph such that $ |Z _{n-1}s _{n}| \leq |Z _{n-1}|$ only if $ s _{n} $ corresponds to a vertex in the clique. Likewise, for each word $ w _{n} $, by considering $ s _{n} ^{-1} $ and $ w _{n} ^{-1} $. we have that a unique well-defined clique such that $ |w _{n}| \leq  |s _{n} w _{n}| $ only if $ s _{n} $ corresponds to a vertex in this clique. We call these cliques the \emph{terminal clique} of $ Z _{n} $ and the \emph{initial clique} of $ w _{n} $, respectively.



%	  For a word $ w _{n} $ in the RAAG, write $ w _{n} = a ^{j _{1}} _{i _{1}} ... a ^{j _{k}}_{i _{k}} $ where no two adjacent $ a _{i _{k}} $'s come from the same generator, and $ a _{i _{k}} $ and $ a _{i _{m}} $ are different if each of $ a _{i _{k}}, a _{i _{k+1}}, ..., a _{i _{m}} $ commute with $ a _{i _{k}} $. Starting from $ a _{i _{1}} $, consider the longest subword for which all $ a _{i _{1}}, ..., a _{i _{k}} $ commute. The corresponding subgraph of the connection graph is the \emph{initial clique} of $ w _{n} $. We define the \emph{terminal clique} likewise, but starting from the last letter in the word instead.

	  First consider the probability of adding another pivotal time. There are at most three ways for this to avoid adding another pivotal time:

	  \begin{enumerate} 
		  \item If $ s _{n} $ corresponds to a vertex in the initial clique of $ w _{n} $. For example, $ s _{n} = a ^{-1} _{1} $ and $ w _{n} = a _{1}  $.
		  \item If $ s _{n} $ corresponds to a vertex in the terminal clique of $ Z_{n-1}$. For example, if $ Z _{n-1} = a _{1} a _{2} $ and $ s _{n} = a _{1} ^{-1} $.
		  \item If $ s _{n} $ corresponds to a vertex which is adjacent to both the initial clique of $ Z _{n-1} $ and the terminal clique of $ w _{n} $. For example, if $ Z _{n-1} = a _{2} ^{2} $, $ s _{n} = a _{1} $, and $ w _{n} = a _{2} ^{-2} $.
		    \end{enumerate} As our graph is a cycle of length greater than $ 6 $, then each clique has size at most 2, and for any two cliques $ K _{1}, K _{2} $ there is at most one vertex in $ C _{d} \setminus (K _{1}, K _{2})  $ adjacent to both. Therefore there are at most $ 5 $ possible vertices in the connection graph for which we don't add another pivotal time.

	  As our measure is uniform on our symmetric generating set, this means that $ \mathbb{P} (A _{n+1} = A _{n} +1) \geq \frac{2d-2\cdot 5}{2d} = \frac{d-5}{d} $.

	  Now consider the probability of backtracking (i.e. the probability that $ A _{n+1} = A _{n} - q $ for some $ q>0 $). 

	  First consider the probability that $ A _{n+1} \leq A _{n} -1 $. This first requires that our choice of $ s _{n} $ does not cause the number of pivotal times to increase, with happens with probability at most $ \frac{5}{d} $. Now let $ k = \max \{P _{n-1}\}  $ be the last pivotal time. For this time to no longer be pivotal, it requires that the vertex corresponding to $ s _{k} $ is now adjacent to the initial clique of $ w _{k}...s _{n}w _{n} $. As $ |s _{n}| = 1 $ and $ w _{k}, w _{n} $ are nontrivial, then this can occur with probability at most $ \frac{4}{d-5} $. The 4 comes from requiring adjacency to the initial clique, and the $ d-5 $ comes from requiring that $ s _{k} $ was already pivotal to begin with. Therefore we lose 1 pivotal time with probability at most $ \frac{5}{d}\cdot  \frac{4}{d-5}  $.

	  By the same argument, we have probability at most $ \frac{5}{d} \left(\frac{4}{d-5} \right) ^{j}$ of losing $ j $ pivotal times, for $ j > 0 $.

	  Therefore $ A _{n+1} $ stochastically dominates $ A _{n} + U _{n} $ where $ \mathbb{P} (U=1) = \frac{d-5}{d} $ and $ \mathbb{P}(U \leq -j) = \frac{5}{d} \left( \frac{4}{d-5} \right) ^{j} $ for $ j \geq 0 $. This has expectation $ \frac{(d-5)(d-14)}{d(d-9)} $, which is positive when $ d > 14 $. As the random variable $ U $ has exponential tail, we know by the large deviations bound that there exists some $ \kappa > 0 $ such that $ \mathbb{P} (|Z _{n}| \leq \kappa n) \leq e ^{-\kappa n} $.

	  As the random variables $ U = U(d) $ are increasing with $ d $, then we can choose $ \kappa(14) $ as a universal bound.

\end{document}

